
The headers of a test are not just there for decoration; three properties are enforced
when a test, say, \texttt{(*\$X foo} is compiled, where \texttt{X} is
\texttt{T}, \texttt{R}, \texttt{Q},... :

\begin{itemize}

\item  \texttt{foo} exists; that is to say, it is defined in the scope of the module where the testappears -- though one can play with pragmas to relax this condition somewhat. At the very
least, it has to be defined \emph{somewhere}. Failure to conform results in an \texttt{Error: Unbound
value foo}.

\item \texttt{foo} is referenced in \emph{each statement} of the test:
  for \texttt{T} and \texttt{Q}, that means "each line". For
  \texttt{R}, that means "once somewhere in the test's body". Failure
  to conform results in a \texttt{Warning 26: unused variable foo},
  which will be treated as an error if \texttt{-warn-error +26} is
  passed to the compiler. It goes without saying that this is warmly
  recommended.

\item  the test possesses at least one statement.
\end{itemize}

Those two conditions put together offer a strong guarantee that, if a function is
referenced in a test header, then it is actually tested at least once. The list of
functions referenced in the headers of extracted tests is written by \qtest{} into
\texttt{qtest.targets.log}. Each line is of the form

\begin{verbatim}
foo.ml   42    foo
\end{verbatim}

where \texttt{foo.ml} is the file in which the test appears, as passed to \texttt{extract},
 and \texttt{42} is
the line number where the test pragma appears in \texttt{foo.ml}. Note that a same function can be
listed several times for the same source file, if several tests involve it (say, two times
if it has both a simple test and a random one). The exact number of statements involving
\texttt{foo} in each test is currently not taken into account in the logs.
