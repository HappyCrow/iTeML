
The vast majority of test cases tend to involve the equality of
two expressions; using simple tests, one would write something like:

\begin{verbatim}
(*$T foo
  foo 1 ( * ) [4;5] = foo 3 ( * ) [1;5;2]
*)
\end{verbatim} 

While this certainly works, the failure report for such a test does not convey
any useful information besides the simple fact that the test failed. Wouldn't it be
nice if the report also mentioned the values of the left-hand side and the
right-hand side ? Yes it would, and specialised equality tests provide such functionality,
at the cost of a little bit of boilerplate code. The bare syntax is:

\begin{verbatim}
(*$= <header>
  <lhs> <rhs>
  ...
*)
\end{verbatim}

However, used bare, an equality test will not provide much more information than a simple test:
just a laconic ``not equal''. In order for the values to be printed, a ``value printer''
must be specified for the test. A printer is a function of type $\alpha\to \textsf{string}$,
where $\alpha$ is the type of the expressions on both side of the equality.
To pass the printer to the test, we use \emph{parameter injection} (cf. Section \ref{sec:param});
equality tests have an optional argument \texttt{printer} for this purpose.
In our example, we have $\alpha = \textsf{int}$, so the test becomes simply:

\begin{verbatim}
(*$= foo & ~printer:string_of_int
  (foo 1 ( * ) [4;5]) (foo 3 ( * ) [1;5;2])
*)
\end{verbatim}

The failure report will now be more explicit, saying \texttt{expected: 20 but got: 30
}.

