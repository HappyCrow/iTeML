\documentclass[a4paper,12pt]{article}
\usepackage[utf8]{inputenc}
\usepackage{url, web}
%%%%%%%%%%%%%%%%%%%%%%%%%%%%%%%
\renewcommand{\cuttingunit}{subsection}


\nc\qtest{\textbf{q\textit{test}}}
\nc\bat{\lnk{Batteries}{http://batteries.forge.ocamlcore.org}}
\nc\ounit{\lnk{oUnit}{http://ounit.forge.ocamlcore.org}}
%opening
\title{\Large{\qtest}\footnote{
  This page was last updated on \today, \thehour:\theminute:\thesecond.}\\
Quick Unit Tests for \OCaml\\
version 2.0
}
\author{Vincent \textsc{Hugot} \& the \bat\ team}

\begin{document}
\maketitle
\tableofcontents

\section{Introduction}

In a nutshell, \qtest{} is a small program which reads \texttt{.ml} and \texttt{.mli} source file and
extracts inline \ounit{} unit tests from them. It is used internally by the OCaml
\bat{} project, and is shipped with it as of version 2.0, but it does not depend on
it and can be compiled and used independently.

Browse its code in the \bat\ Github
\lnk{repository}{https://github.com/ocaml-batteries-team/batteries-included/tree/master/qtest}.



\section{Using \qtest{} : a Quick, Simple Example}

Say that you have a file \texttt{foo.ml}, which contains the implementation of your new, shiny
function \texttt{foo}.

\begin{verbatim}
let rec foo x0 f = function
  [] -> 0 | x::xs -> f x (foo x0 f xs)
\end{verbatim}


Maybe you don't feel confident about that code; or maybe you do, but you know that the
function might be re-implemented less trivially in the future and want to prevent
potential regressions. Or maybe you simply think unit tests are good practice anyway. In
either case, you feel that building a separate test suite for this would be overkill.
Using \qtest{}, you can immediately put simple unit tests in comments near \texttt{foo}, for
instance:

\begin{verbatim}
(*$T foo
  foo  0 ( + ) [1;2;3] = 6
  foo  0 ( * ) [1;2;3] = 0
  foo  1 ( * ) [4;5]   = 20
  foo 12 ( + ) []      = 12
*)
\end{verbatim} 

the syntax is simple: \texttt{(*\$} introduces a \qtest{} "pragma", such as \texttt{T} in this case. \texttt{T}
is by far the most common and represents a "simple" unit test. \texttt{T} expects a "header",
which is most of the time simply the name of the function under test, here \texttt{foo}.
Following that, each line is a "statement", which must evaluate to true for the test to
pass. Furthermore, \texttt{foo} must appear in each statement.

Now, in order to execute those tests, you need to extract them; this is done with the
\qtest{} executable. The command

\Oconsole
\begin{verbatim}
$ qtest -o footest.ml extract foo.ml
Target file: \texttt{footest.ml'. Extraction : }foo.ml' Done.
\end{verbatim} 

will create a file \texttt{footest.ml}; it's not terribly human-readable, but you can see that it
contains your tests as well as some \ounit{} boilerplate. Now you need to compile the
tests, for instance with \texttt{ocamlbuild}, and assuming \ounit{} was installed for \texttt{ocamlfind}.
    
\begin{verbatim}
$ ocamlbuild -cflags -warn-error,+26 -use-ocamlfind -package oUnit \
    footest.native
Finished, 10 targets (1 cached) in 00:00:00.
\end{verbatim}

Note that the \texttt{-cflags -warn-error,+26} is not indispensable but strongly recommended. Its
function will be explained in more detail in the more technical sections of this
documentation, but roughly it makes sure that if you write a test for \texttt{foo}, via
\texttt{(*\$T foo} for instance, then \texttt{foo} is *actually* tested by each statement
-- the tests won't compile if not.

\textbf{Important note:} in order for this to work, \texttt{ocamlbuild} must know where to find
\texttt{foo.ml}; if \texttt{footest.ml} is not in the same directory, you must make provisions to that
effect. If \texttt{foo.ml} needs some specific flags in order to compile, they must also be
passed.


Now there only remains to run the tests:

\begin{verbatim}
$ ./footest.native
..FF
==============================================================================
Failure: qtest:0:foo:3:foo.ml:10

OUnit: foo.ml:10::>  foo 12 ( + ) [] = 12
------------------------------------------------------------------------------
==============================================================================
Failure: qtest:0:foo:2:foo.ml:9

OUnit: foo.ml:9::>  foo 1 ( * ) [4;5] = 20
------------------------------------------------------------------------------
Ran: 4 tests in: 0.00 seconds.
FAILED: Cases: 4 Tried: 4 Errors: 0 Failures: 2 Skip:0 Todo:0
\end{verbatim} 

Oops, something's wrong... either the tests are incorrect of \texttt{foo} is. Finding and fixing
the problem is left as an exercise for the reader. When this is done, you get the expected

\begin{verbatim}
$ ./footest.native
....
Ran: 4 tests in: 0.00 seconds.
\end{verbatim}

\textbf{Tip:} those steps are easy to automate, for instance with a small shell script:


\begin{verbatim}
set -e # stop on first error
qtest -o footest.ml extract foo.ml
ocamlbuild -cflags -warn-error,+26 -use-ocamlfind -package oUnit footest.native
./footest.native
\end{verbatim} 
\Overbatim
    
\section{More \qtest{} Pragmas}

\subsection{Different Kinds of Tests}
    
\subsubsection{Simple Tests: \textttl{T} for "Test"}

The most common kind of tests is the simple test, an example of which is given above. It
is of the form

\begin{verbatim}
(*$T <header>
  <statement>
  ...
*)
\end{verbatim}

where each *statement* must be a boolean OCaml expression involving the function (or
functions, as we will see when we study headers) referenced in the *header*.
The overall test is considered successful if each *statement* evaluates to \textbf{true}. Note
that the "close comment" \texttt{*)} must appear on a line of its own.

\textbf{Tip:} if a statement is a bit too long to fit on one line, if can be broken using a
backslash (\texttt{\backslash}), immediately followed by the carriage return. This also applies to
randomised tests.

\subsubsection{Randomized Tests: \textttl{Q} for "Quickcheck"}

Quickcheck is a small library useful for randomized unit tests. Using it is a bit more
complex, but much more rewarding than simple tests.

\begin{verbatim}
(*$Q <header>
  <generator> (fun <generated value> -> <statement>)
  ...
*)
\end{verbatim}

Let us dive into an example straightaway:

\begin{verbatim}
(*$Q foo
  Q.small_int (fun i-> foo i (+) [1;2;3] = List.fold_left (+) i [1;2;3])
*)
\end{verbatim}

The Quickcheck module is accessible simply as \emph{Q} within inline tests; \texttt{small\_int} is a
generator, yielding a random, small integer. When the test is run, each statement will be
evaluated for a large number of random values -- 100 by default. Running this test for the
above definition of foo catches the mistake easily:

{\Oconsole\begin{verbatim}
law foo.ml:14::>  Q.small_int (fun i-> foo i (+) [1;2;3]
    = List.fold_left (+) i [1;2;3])
failed for 2
\end{verbatim} }

Note that the random value for which the test failed is provided by the error message --
here it is 2. It is also possible to generate several random values simultaneously using
tuples. For instance


\begin{verbatim}
(Q.pair Q.small_int (Q.list Q.small_int)) \
  (fun (i,l)-> foo i (+) l = List.fold_left (+) i l)
\end{verbatim} 
    
will generate both an integer and a list of small integers randomly. A failure will then
look like

{\Oconsole\begin{verbatim}
law foo.ml:15::>  (Q.pair Q.small_int (Q.list Q.small_int))
    (fun (i,l)-> foo i (+) l = List.fold_left (+) i l)
failed for (727, [4; 3; 6; 1; 788; 49])
\end{verbatim}}

\textbf{Available Generators:}

\begin{itemize}
\item  \underline{Simple generators:}\\
\texttt{unit}, \texttt{bool}, \texttt{float}, \texttt{pos\_float},
\texttt{neg\_float}, \texttt{int}, \texttt{pos\_int}, \texttt{small\_int},
\texttt{neg\_int}, \texttt{char}, \texttt{printable\_char}, \texttt{numeral\_char},
\texttt{string}, \texttt{printable\_string},
\texttt{numeral\_string}

\item \underline{Structure generators:}\\
\texttt{list} and \texttt{array}. They take one generator as their argument. For instance
\texttt{(Q.list Q.neg\_int)} is a generator of lists of (uniformly taken) negative integers.

\item \underline{Tuples generators:}\\
\texttt{pair} and \texttt{triple} are respectively binary and ternary. See above for an example of
\texttt{pair}.

\item \underline{Size-directed generators:}\\
\texttt{string}, \texttt{numeral\_string}, \texttt{printable\_string}, \texttt{list}
and \texttt{array} all have \texttt{*\_of\_size}
variants that take the size of the structure as their first argument.
\end{itemize}

\subsubsection{Raw \ounit{} Tests: \textttl{R} for "Raw"}

When more specialised test pragmas are too restrictive, for instance if the test is too
complex to reasonably fit on one line, then one can use raw \ounit{} tests.

\begin{verbatim}
(*$R <header>
  <raw oUnit test>...
  ...
*)
\end{verbatim} 

Here is a small example, with two tests stringed together:

\begin{verbatim}
(*$R foo
  let thing = foo  1 ( * )
  and li = [4;5] in
  assert_bool "something_witty" (thing li = 20);
  assert_bool "something_wittier" (foo 12 ( + ) [] = 12)
*)
\end{verbatim}

Note that if the first assertion fails, the second will not be executed; so stringing two
assertions in that mode is different in that respect from doing so under a \texttt{T} pragma, for
instance.

That said, raw tests should only be used as a last resort; for instance you don't
automatically get the source file and line number when the test fails. If \texttt{T} and \texttt{Q} do
not satisfy your needs, then it is *probably* a hint that the test is a bit complex and,
maybe, belongs in a separate test suite rather than in the middle of the source code.

\subsubsection{Exception-Throwing Tests: \textttl{E} for "Exception"}

... not implemented yet...

The current usage is to use \texttt{(*\$T} and the following pattern for function \texttt{foo} and
exception \texttt{Bar}:

\begin{verbatim}
try ignore (foo x); false with Bar -> true
\end{verbatim} 

If your project uses *Batteries* and no pattern-matching is needed, then you can also use
the following, sexier pattern:

\begin{verbatim}
Result.(catch foo x |> is_exn Bar)
\end{verbatim} 

\subsection{Manipulation Pragmas}

Not all qtest pragmas directly translate into tests; for non-trivial projects, sometimes a
little boilerplate code is needed in order to set the tests up properly. The pragmas which
do this are collectively called "manipulation pragmas"; they are described in the next
section.

\subsubsection{Opening Modules: *open* Pragma \textttl{<...>} and \textttl{-{}-preamble} Option}

The tests should have access to the same values as the code under test; however the
generated code for, say, \texttt{foo.ml} does not actually live inside that file. Therefore some
effort must occasionally be made to synchronise the code's environment with the tests'.
There are three main usecases where you might want to open modules for tests:

\begin{itemize}
 \item \textbf{Project-Wide Global Open:}\\
It may happen that \emph{every single file} in your project
opens a given module. This is the case for \bat, for instance, where every module
opens \texttt{Batteries}. In that case simply use the \texttt{--preamble} switch. For instance,
\begin{verbatim}
qtest --preamble "open Batteries;;"  extract mod1.ml mod2.ml ... modN.ml
\end{verbatim} 
Note that you could insert arbitrary code using this switch.

\item \textbf{Global Open in a File:}\\
 Now, let's say that \texttt{foo.ml} opens \texttt{Bar} and \texttt{Baz}; you want
the tests in \texttt{foo.ml} to open them as well. Then you can use the *open* pragma in its
\emph{global} form:
\begin{verbatim}
(*$< Bar, Baz >*)
\end{verbatim} 
The modules will be open for every test in the same \texttt{.ml} file, and following the pragma.
However, in our example, you will have a duplication of code between the "open" directives
of \texttt{foo.ml}, and the *open* pragma of \qtest{}, like so:
\begin{verbatim}
open Bar;; open Baz;;
(*$< Bar, Baz >*)
\end{verbatim} 
It might therefore be more convenient to use the *code injection* pragma (see next
section) for that purpose, so you would write instead:
\begin{verbatim}
(*${*) open Bar;; open Baz;; (*$}*)
\end{verbatim} 
The code between that special markup will simply be duplicated into the tests. The two
methods are equivalent, and the second one is recommended, because it reduces the chances
of an impedance mismatch between modules open for `\texttt{foo.ml}` and its tests. Therefore, the
global form of the *open* pragma should preferentially be reserved for cases where you
\emph{want} such a mismatch. For instance, if you have special modules useful for tests but
useless for the main code, you can easily open then for the tests alone using the pragma.

\item \textbf{Local Open for a Submodule:}\\
 Let's say we have the following \texttt{foo.ml}:
\begin{verbatim}
let outer x = <something>

module Submod = struct
  let inner y = 2*x
  (*$T inner
    inner 2 = 4
  *)
end
\end{verbatim} 
That seems natural enough... but it won't work, because \qtest{} is not actually aware that
the test is "inside" Submod (and making it aware of that would be very problematic). In
fact, so long as you use only test pragmas (ie. no manipulation pragma at all), the
positions and even the order of the tests -- respective to definitions or to each other --
are unimportant, because the tests do not actually live in \texttt{foo.ml}. So we need to open
Submod manually, using the *local* form of the *open* pragma:
\begin{verbatim}
module Submod = struct (*$< Submod *)
  let inner y = 2*x
  (*$T inner
    inner 2 = 4
  *)
end (*$>*)
\end{verbatim} 
Notice that the \texttt{<...>} have simply been split in two, compared to the global form. Of
course, you *could* also forgo that method entirely and do this:
\begin{verbatim}
module Submod = struct
  let inner y = 2*x
  (*$T &
    Submod.inner 2 = 4
  *)
end
\end{verbatim} 
... but it is impractical and you are *forced* to use an empty header because qualified
names are not acceptable as headers. The first method is therefore *strongly* recommended.
\end{itemize}


\subsubsection{Code Injection Pragma: \textttl{\{...\}}}



TODO: ocamldoc comments that define unit tests from the offered examples

\section{Technical Considerations and Other Details}

What has been said above should suffice to cover at least 90\% of use-cases for \qtest{}.
This section concerns itself with the remaining 10\%.

\subsection{Function Coverage}

The headers of a test are not just there for decoration; three properties are enforced
when a test, say, \texttt{(*\$X foo} is compiled, where \texttt{X} is
\texttt{T}, \texttt{R}, \texttt{Q},... :

\begin{itemize}

\item  \texttt{foo} exists; that is to say, it is defined in the scope of the module where the testappears -- though one can play with pragmas to relax this condition somewhat. At the very
least, it has to be defined *somewhere*. Failure to conform results in an `Error: Unbound
value foo`.

\item  \texttt{foo} is referenced in *each statement* of the test: for \texttt{T} and \texttt{Q}, that means "each
line". For \texttt{R}, that means "once somewhere in the test's body". Failure to conform results
in a \texttt{Warning 26: unused variable foo}, which will be treated as an error if `-warn-error
+26` is passed to the compiler. It goes without saying that this is warmly recommended.

\item  the test possesses at least one statement.
\end{itemize}

Those two conditions put together offer a strong guarantee that, if a function is
referenced in a test header, then it is actually tested at least once. The list of
functions referenced in the headers of extracted tests is written by \qtest{} into
\texttt{qtest.targets.log}. Each line is of the form

\begin{verbatim}
foo.ml   42    foo
\end{verbatim}

where \texttt{foo.ml} is the file in which the test appears, as passed to \texttt{extract} and \texttt{42} is
the line number where the test pragma appears in foo.ml. Note that a same function can be
listed several times for the same source file, if several tests involve it (say, two times
if it has both a simple test and a random one). The exact number of statements involving
\texttt{foo} is currently not taken into account in the logs.

\subsection{Headers and Metaheaders}

The informal definition of headers given in the above was actually a simplification. In
this section we explore to syntaxes available for headers.

\subsubsection{Aliases}

Some functions have exceedingly long names. Case in point :

\begin{verbatim}
let rec pretentious_drivel x0 f = function [] -> x0
  | x::xs -> pretentious_drivel (f x x0) f xs
\end{verbatim} 
  
\begin{verbatim}
(*$T pretentious_drivel
  pretentious_drivel 1 (+) [4;5] = foo 1 (+) [4;5]
  ... pretentious_drivel of this and that...
*)
\end{verbatim}

The constraint that each statement must fit on one line does not play well with very long
function names. Furthermore, you *known* which function is being tested, it's right there
is the header; no need to repeat it a dozen times. Instead, you can define an *alias*, and
write equivalently:

\begin{verbatim}
(*$T pretentious_drivel as x
  x 1 (+) [4;5] = foo 1 (+) [4;5]
  ... x of this and that...
*)
\end{verbatim}

... thus saving many keystrokes, thereby contributing to the preservation of the
environment. More seriously, aliases have uses beyond just saving a few keystrokes, as we
will see in the next sections.
    
\subsubsection{Mutually Tested Functions}

Most of the time, a test only pertains to one function. There are times, however, when one
wishes to test two functions -- or more -- at the same time. For instance

\begin{verbatim}
let rec even = function 0 -> true
  | n -> odd (pred n)
and odd = function 0 -> false
  | n -> even (pred n)
\end{verbatim} 

Let us say that we have the following test:

\begin{verbatim}
(*$Q <header>
  Q.small_int (fun n-> odd (abs n+3) = even (abs n))
*)
\end{verbatim}

It involves both \texttt{even} and \texttt{odd}. That question is: "what is a proper header for this
test"? One could simply put "even", and thus it would be referenced as being tested in the
logs, but \texttt{odd} would not, which is unfair. Putting "odd" is symmetrically unfair. The
solution is to put both, separated by a semi-colon:
    
\begin{verbatim}
(*$Q even; odd
\end{verbatim}

That way *both* functions are referenced in the logs:

\begin{verbatim}
    foo.ml   37    even
    foo.ml   37    odd
\end{verbatim}

and of course the compiler enforces that both of them are actually referenced in each
statement of the test. Of course, each of them can be written under alias, in which case
the header could be \texttt{even as x; odd as y}.

\subsubsection{Testing Functions by the Dozen}

Let us come back to our functions \texttt{foo} (after correction) and \texttt{pretentious\_drivel}, as
defined above.


\begin{verbatim}
let rec foo x0 f = function
  [] -> x0 | x::xs -> f x (foo x0 f xs)
  
let rec pretentious_drivel x0 f = function [] -> x0
  | x::xs -> pretentious_drivel (f x x0) f xs
\end{verbatim} 

You will not have failed to notice that they bear more than a passing resemblance to one
another. If you write tests for one, odds are that the same test could be useful verbatim
for the other. This is a very common case when you have closely related functions, or even
several *implementations* of the same function, for instance the old, slow, naïve,
trustworthy one and the new, fast, arcane, highly optimised version you have just written.
The typical case is sorting routines, of which there are many flavours.

For our example, recall that we have the following test for \texttt{foo}:

\begin{verbatim}
(*$Q foo
  (Q.pair Q.small_int (Q.list Q.small_int)) \
    (fun (i,l)-> foo i (+) l = List.fold_left (+) i l)
*)
\end{verbatim}

The same test would apply to \texttt{pretentious\_drivel}; you could just copy-and-paste the test
and change the header, but it's not terribly elegant. Instead, you can just just add the
other function to the header, separating the two by a comma, and defining an alias:

\begin{verbatim}
(*$Q foo, pretentious_drivel as x
  (Q.pair Q.small_int (Q.list Q.small_int)) \
  (fun (i,l)-> x i (+) l = List.fold_left (+) i l)
*)
\end{verbatim}

This same test will be run once for \texttt{x = foo}, and once for \texttt{x = pretentious\_drivel}.
Actually, you need not define an alias: if the header is of the form
    
\begin{verbatim}
(*$Q foo, pretentious_drivel
\end{verbatim}

then it is equivalent to

\begin{verbatim}
(*$Q foo, pretentious_drivel as foo
\end{verbatim}

so you do not need to alter the body of the test if you subsequently add new functions. A
header which combines more than one "version" of a function in this way is called a
\emph{metaheader}.

\subsubsection{Metaheaders Unleashed}

All the constructs above can be combined without constraints: the grammar is as follows:

\begin{verbatim}
    Metaheader  ::=   Binding {";" Binding}
    Binding     ::=   Functions [ "as" ID ]
    Functions   ::=   ID {"," ID}
    ID          ::=   (*OCaml lower-case identifier*)
\end{verbatim}

\subsection{Warnings and Exceptions Thrown by \qtest{}}

\Oconsole
\begin{verbatim}
Fatal error: exception Failure("Unrecognised qtest pragma: ` T foo'")
\end{verbatim}

You have written something like \texttt{(*\$ T foo}; there must not be any space
between \texttt{(*\$} and
the pragma.

\begin{verbatim}
Warning: likely qtest syntax error: `(* $T foo'. Done.
\end{verbatim}

Self-explanatory; if \texttt{\$} is the first real character of a comment, it's likely a mistyped
qtest pragma. This is only a warning though.

\begin{verbatim}
Fatal error: exception Core.Bad_header_char("M", "Mod.foo")
\end{verbatim}

You have used a qualified name in a header, for instance \texttt{(*\$T Mod.foo}. You cannot do
that, the name must be unqualified and defined under the local scope. Furthermore, it must
be public, unless you have used pragmas to deal with private functions.

\begin{verbatim}
Error: Comment not terminated
Fatal error: exception Core.Unterminated_test(_, 0)
\end{verbatim}

Most probably, you forgot the comment-closing \texttt{*)} to close some test.

\begin{verbatim}
Fatal error: exception Failure("runaway test body terminator: n))*)")
\end{verbatim}

The comment-closing \texttt{*)} must be on a line of its own; or, put another way, every
statement must be ended by a line break.

\subsection{\qtest{} Command-Line Options}

\begin{verbatim}
$ qtest --help

** qtest (qtest)
USAGE: qtest [options] extract <file.mli?>...

OPTIONS:
--output <file.ml>    (-o) def: standard output
  Open or create a file for output; the resulting file will be an OCaml
  source file containing all the tests.

--preamble <string>   (-p) def: empty
  Add code to the tests' preamble; typically this will be an instruction
  of the form 'open Module;;'


--help          Displays this help page and stops
\end{verbatim} 

\end{document}
