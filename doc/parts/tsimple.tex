
The most common kind of tests is the simple test, an example of which is given above. It
is of the form

\begin{verbatim}
(*$T <header>
  <statement>
  ...
*)
\end{verbatim}

where each \emph{statement} must be a boolean \OCaml{} expression involving the function (or
functions, as we will see when we study headers) referenced in the \emph{header}.
The overall test is considered successful if each \emph{statement} evaluates to \textbf{true}. Note
that the "close comment" \texttt{*)} must appear on a line of its own.

\textbf{Tip:} if a statement is a bit too long to fit on one line, if can be broken using a
backslash (\texttt{\backslash}), immediately followed by the carriage return. This also applies to
randomised tests.