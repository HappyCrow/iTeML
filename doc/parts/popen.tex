
The tests should have access to the same values as the code under test; however the
generated code for \texttt{foo.ml} does not actually live inside that file. Therefore some
effort must occasionally be made to synchronise the code's environment with the tests'.
There are three main usecases where you might want to open modules for tests:

\begin{itemize}
 \item \textbf{Project-Wide Global Open:}\\
It may happen that \emph{every single file} in your project
opens a given module. This is the case for \bat, for instance, where every module
opens \texttt{Batteries}. In that case simply use the \texttt{--preamble} switch. For instance,
\begin{verbatim}
qtest --preamble "open Batteries;;"  extract mod1.ml mod2.ml ... modN.ml
\end{verbatim}
Note that you could insert arbitrary code using this switch.

\item \textbf{Global Open in a File:}\\
 Now, let's say that \texttt{foo.ml} opens \texttt{Bar} and \texttt{Baz}; you want
the tests in \texttt{foo.ml} to open them as well. Then you can use the \emph{open} pragma in its
\emph{global} form:
\begin{verbatim}
(*$< Bar, Baz >*)
\end{verbatim}
The modules will be open for every test in the same \texttt{.ml} file, and following the pragma.
However, in our example, you will have a duplication of code between the "open" directives
of \texttt{foo.ml}, and the \emph{open} pragma of \qtest{}, like so:
\begin{verbatim}
open Bar;; open Baz;;
(*$< Bar, Baz >*)
\end{verbatim}
It might therefore be more convenient to use the \emph{code injection} pragma (see next
section) for that purpose, so you would write instead:
\begin{verbatim}
(*${*) open Bar;; open Baz;; (*$}*)
\end{verbatim}
The code between that special markup will simply be duplicated into the tests. The two
methods are equivalent, and the second one is recommended, because it reduces the chances
of an impedance mismatch between modules open for `\texttt{foo.ml}` and its tests. Therefore, the
global form of the \emph{open} pragma should preferentially be reserved for cases where you
\emph{want} such a mismatch. For instance, if you have special modules useful for tests but
useless for the main code, you can easily open then for the tests alone using the pragma.

\item \textbf{Local Open for a Submodule:}\\
 Let's say we have the following \texttt{foo.ml}:
\begin{verbatim}
let outer x = <something>

module Submod = struct
  let inner y = 2*x
  (*$T inner
    inner 2 = 4
  *)
end
\end{verbatim}
That seems natural enough... but it won't work, because \qtest{} is not actually aware that
the test is "inside" Submod (and making it aware of that would be very problematic). In
fact, so long as you use only test pragmas (ie. no manipulation pragma at all), the
positions and even the order of the tests -- respective to definitions or to each other --
are unimportant, because the tests do not actually live in \texttt{foo.ml}. So we need to open
Submod manually, using the \emph{local} form of the \emph{open} pragma:
\begin{verbatim}
module Submod = struct (*$< Submod *)
  let inner y = 2*x
  (*$T inner
    inner 2 = 4
  *)
end (*$>*)
\end{verbatim}
Notice that the \texttt{<...>} have simply been split in two, compared to the global form.
The effect of that construct is that Submod will be open for every test between
\texttt{(*\$< Submod *)} and \texttt{(*\$>*)}.
Of course, you \emph{could} also forgo that method entirely and do this:
\begin{verbatim}
module Submod = struct
  let inner y = 2*x
  (*$T &
    Submod.inner 2 = 4
  *)
end
\end{verbatim}
... but it is impractical and you are \emph{forced} to use an empty header because qualified
names are not acceptable as headers. The first method is therefore \emph{strongly} recommended.
\end{itemize}
