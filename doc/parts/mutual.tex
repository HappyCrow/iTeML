
Most of the time, a test only pertains to one function. There are times, however, when one
wishes to test two functions -- or more -- at the same time. For instance

\begin{verbatim}
let rec even = function 0 -> true
  | n -> odd (pred n)
and odd = function 0 -> false
  | n -> even (pred n)
\end{verbatim}

Let us say that we have the following test:

\begin{verbatim}
(*$Q <header>
  Q.small_int (fun n-> odd (abs n+3) = even (abs n))
*)
\end{verbatim}

It involves both \texttt{even} and \texttt{odd}. That question is: "what is a proper header for this
test"? One could simply put "even", and thus it would be referenced as being tested in the
logs, but \texttt{odd} would not, which is unfair. Putting "odd" is symmetrically unfair. The
solution is to put both, separated by a semi-colon:

\begin{verbatim}
(*$Q even; odd
\end{verbatim}

That way \emph{both} functions are referenced in the logs:

\begin{verbatim}
    foo.ml   37    even
    foo.ml   37    odd
\end{verbatim}

and of course the compiler enforces that both of them are actually referenced in each
statement of the test. Of course, each of them can be written under alias, in which case
the header could be \texttt{even as x; odd as y}.
