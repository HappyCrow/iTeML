
Let us come back to our functions \texttt{foo} (after correction) and \texttt{pretentious\_drivel}, as
defined above.


\begin{verbatim}
let rec foo x0 f = function
  [] -> x0 | x::xs -> f x (foo x0 f xs)

let rec pretentious_drivel x0 f = function [] -> x0
  | x::xs -> pretentious_drivel (f x x0) f xs
\end{verbatim}

You will not have failed to notice that they bear more than a passing resemblance to one
another. If you write tests for one, odds are that the same test could be useful verbatim
for the other. This is a very common case when you have closely related functions, or even
several \emph{implementations} of the same function, for instance the old, slow, naïve,
trustworthy one and the new, fast, arcane, highly optimised version you have just written.
The typical case is sorting routines, of which there are many flavours.

For our example, recall that we have the following test for \texttt{foo}:

\begin{verbatim}
(*$Q foo
  (Q.pair Q.small_int (Q.list Q.small_int)) \
    (fun (i,l)-> foo i (+) l = List.fold_left (+) i l)
*)
\end{verbatim}

The same test would apply to \texttt{pretentious\_drivel}; you could just copy-and-paste the test
and change the header, but it's not terribly elegant. Instead, you can just just add the
other function to the header, separating the two by a comma, and defining an alias:

\begin{verbatim}
(*$Q foo, pretentious_drivel as x
  (Q.pair Q.small_int (Q.list Q.small_int)) \
  (fun (i,l)-> x i (+) l = List.fold_left (+) i l)
*)
\end{verbatim}

This same test will be run once for \texttt{x = foo}, and once for \texttt{x = pretentious\_drivel}.
Actually, you need not define an alias: if the header is of the form

\begin{verbatim}
(*$Q foo, pretentious_drivel
\end{verbatim}

then it is equivalent to

\begin{verbatim}
(*$Q foo, pretentious_drivel as foo
\end{verbatim}

so you do not need to alter the body of the test if you subsequently add new functions. A
header which combines more than one "version" of a function in this way is called a
\emph{metaheader}.